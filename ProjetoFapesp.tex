\documentclass[a4paper,12pt]{article}
\usepackage[utf8]{inputenc}
%\usepackage[latin1]{inputenc}
\usepackage[english,brazil]{babel}
\usepackage{verbatim}
\usepackage{setspace}
% \usepackage{harvard}
\usepackage[]{graphicx}
\usepackage{caption}
\usepackage{url}
\usepackage{subfigure}
%\usepackage{wasysym} - Se usá-lo, não usar amsmath
\hyphenation{heu-ris-ti-cas}
\usepackage{algorithm}
\usepackage{algorithmic}
\usepackage{amsthm}
\usepackage{amsmath}
\usepackage{multirow}
% Referências e hiperlinks sem o "box" vermelho.
\usepackage[hypertexnames=false,pdfborder={0 0 0}]{hyperref}

% Lê a codificação de fonte T1 (font encoding default é 0T1).
\usepackage[T1]{fontenc}

% Equações, símbolos matemáticos e fontes providos pela American Mathematical Society (AMS).
%\usepackage{amssymb}
\usepackage{amssymb,amsfonts}	

%\usepackage[center]{caption2}
% \usepackage[portuguese,algoruled,longend]{algorithm2e}
% \usepackage{algorithmic}
\usepackage{lipsum}
\usepackage[cm]{fullpage}

\setlength{\paperheight}{29.79cm}
\setlength{\paperwidth}{21.1cm}
\setlength{\voffset}{0.32cm}
\setlength{\hoffset}{0.055cm}
\setlength{\oddsidemargin}{0.02cm}
\setlength{\evensidemargin}{-0.605cm}
\setlength{\topmargin}{0.0cm}
\setlength{\headheight}{14.5pt}
\setlength{\headsep}{0.3cm}
\setlength{\textheight}{24.025cm}
\setlength{\textwidth}{15.7cm}
\setlength{\marginparsep}{0.0cm}
\setlength{\marginparwidth}{0.0cm}
\setlength{\marginparpush}{0.0cm}
\setlength{\footskip}{0.0cm}

%\textwidth = 17cm
%\textheight= 22.0cm
%\oddsidemargin = 1.3cm
%\evensidemargin = 0.7cm
%\voffset = -1.0cm
%\hoffset = -2cm
%\belowcaptionskip = 10pt

% Formatação de cabeçalhos.
\usepackage{fancyhdr}
\pagestyle{fancy}
% Apaga as configurações padrão.
\fancyhf{}
\rhead{\thepage}
\renewcommand{\headrulewidth}{0.2pt}

% Lista de siglas e abreviações.
%\usepackage[printonlyused]{acronym}
\usepackage[nolist]{acronym}
% Uso da mesma fonte utilizada no texto nas siglas e abreviações.
\def\bflabel#1{{\textbf{#1}\hfill}}

%% Indentação do primeiro parágrafo dos capítulos.
\usepackage{indentfirst}

%%% Recuo do parágrafo.
\parindent 1.25cm

% Posicionamento de figuras e tabelas.
\usepackage{float}

% As legendas das figuras e tabelas são espacejadas com espaçamento simples
% e os títulos das mesmas são separados da legenda por um hífen.
%\usepackage[labelsep=endash,font=singlespacing]{caption}	

% Formato ABNT
\usepackage[alf,bibjustif]{abntcite} 




% !TeX encoding = UTF-8
\begin{document}

% Lista de Siglas e de Símbolos.
\begin{acronym}[FAPESP-xx] % Colocar aqui a sigla com o maior número de caracteres. (O "-xx" serve só para dar um espaço maior entre as siglas e seus significados.)
	% A
	
	% B
	
	% C
	
	% D
	
	% E
	
	% F
	
	% G

	% H

	
	% I
	
	% J
	
	% K
	
	% L
	
	% M
	
	% N
	
	% O
	
	% P
		
	% Q
	
	% R

	% S
	\acro{SEP}{Sistema Elétrico de Potência}
	\acrodefplural{SEP}[SEPs]{Sistemas Elétricos de Potência}
	
	% T
	
	% U
	
	% W
	
	% V
	
	% X
	
	% Y
	
	% Z

\end{acronym}

% Formato ABNT Autor-Data
\bibliographystyle{abnt-alf}
% Abreviação de apresentação de 4 ou mais autores no texto.
\citeoption{abnt-etal-cite=3}
% Subistituição do nome repetido de um autor por 6 traços underline.
\citeoption{abnt-repeated-author-omit=yes}
 
%%%%%%%%%%%%%%%%%%%%%%%%%%%%%%%%%%%%%% Capa
%%%%%%%%%%%%%%%%%%%%%%%%%%%%%%%%%%%%%%%%%%%%%%%%%%%
\addtocounter{page}{-1}
\thispagestyle{empty}
    
\vspace{-4.0cm}
\begin{figure}[t]
	\centering
		\includegraphics[width=0.31\textwidth]{USP.jpg}
	\label{fig:USP}
\end{figure}

\begin{center}
	\vspace{2 cm}
	\Large{\textbf{UNIVERSIDADE DE SÃO PAULO}}\\[0.5cm]

\large{\bf Escola de Engenharia de São Carlos}\\[0.5cm]

\end{center}

\doublespacing	

\vspace{1.5 cm}
\begin{center}
{\bf PROJETO DE PESQUISA
	
DOUTORADO}
\end{center}

\vspace{1.5 cm}
\begin{center}
	\Large{\textbf{TITLE} }
\end{center}

\singlespacing

\vspace{2 cm}
\begin{center}
	
Aluno: NAME\\[0.5cm]
	
Orientador: NAME
\end{center}

\vspace{1 cm}
%\maketitle
\begin{center}
City, Date
\end{center}

%%%%%%%%%%%%%%%%%%%%%%%%%%%%%%%%%%%%%% Resumo e Abstract %%%%%%%%%%%%%%%%%%%%%%%%%%%%%%%%%%%%%%%%%%%%%%%%%%%
\clearpage
\selectlanguage{brazil}
% Índices --------------------------------------------------------------
\setlength{\parskip}{0mm}

% \maketitle

\onehalfspacing
    
\begin{abstract}
	\noindent TEXT
\end{abstract}

\selectlanguage{english}

\begin{abstract}
	\noindent TEXT

\end{abstract}

\selectlanguage{brazil}

%%%%%%%%%%%%%%%%%%%%%%%%%%%%%%%%%%%%%% Introdução %%%%%%%%%%%%%%%%%%%%%%%%%%%%%%%%%%%%%%%%%%%%%%%%%%%
\newpage
\setlength{\parskip}{3mm}
\doublespacing

\let\clearpage\relax
\section{Introdução}
\label{Introdução}

Inclua síntese da bibliografia fundamental (4 páginas).  Limite de referências de congressos: $30\%$.

Exemplo de uso de citação e acrônimo:

``Normas de qualidade da energia para sistemas de distribuição (parte final do \ac{SEP}), podem ser vistas em \cite{prodist8}''






%%%%%%%%%%%%%%%%%%%%%%%%%%%%%%%%%%%%%% Justificativa %%%%%%%%%%%%%%%%%%%%%%%%%%%%%%%%%%%%%%%%%%%%%%%%%%%
\section{Justificativa}

No máximo, 15 linhas, apresentar as características e conhecimento técnico do beneficiário obtidos, por exemplo, na Graduação, úteis para a realização do projeto. Alguns artigos publicados são bem-vindos como:

%%%%%%%%%%%%%%%%%%%%%%%%%%%%%%%%%%%%%% Objetivos %%%%%%%%%%%%%%%%%%%%%%%%%%%%%%%%%%%%%%%%%%%%%%%%%%%%%%%%%%%%
\section{Objetivos}

\label{sec:Objetivos}

Nesta seção apresenta-se  os objetivos gerais e específicos da pesquisa, conforme segue. (1 - 2 páginas) 

\subsection{Geral}

\subsection{Específicos}

\begin{enumerate}

	\item Texto1;
	
	\item Texto2;
	
\end{enumerate}


%%%%%%%%%%%%%%%%%%%%%%%%%%%%%%%%%%%%%% Cronograma %%%%%%%%%%%%%%%%%%%%%%%%%%%%%%%%%%%%%%%%%%%%%%%%%%%

\section{Plano de Trabalho e Cronograma de Execução}

Nesta seção são detalhados como se pretende realizar cada uma das XX etapas em que o projeto foi dividido e o cronograma de execução planejado. (1 pg.)

\subsection{Plano de Trabalho}

Descreva as etapas .


\subsection{Cronograma de Execução}
Para doutorado 6 semestres, para mestrado 4 semestres: exemplo tabela \ref{tab:1}

\begin{table}[htbp] \centering
	\caption{Cronograma de Execução}
	\label{tab:1}
	\begin{tabular}{l|c|c|c|c|c|c}
		\hline
		\multicolumn{ 1}{c|}{\textbf{Atividades}} & \multicolumn{ 6}{c}{\textbf{Semestres}} \\ 
		\cline{ 2- 7}
		\multicolumn{ 1}{l|}{} & \textbf{1} & \textbf{2} & \textbf{3} & \textbf{4} & \textbf{5} & \textbf{6} \\ \hline
		Levantamento bibliográfico, leitura de artigos, livros, e teses & * & * & * & * & * &  \\ \hline
		Reuniões de orientação & * & * & * & * & * & * \\ \hline
		Cumprimento dos créditos de disciplinas & * & *  &  &  &  &  \\ \hline
		Estudo do problema & * &  &  &  &  &  \\ \hline
		Estudo dos principais métodos de resolução &  & * & * & * &  &  \\ \hline
		Implementação dos modelos matemáticos &  & * & * & * &* &  \\ \hline
		Elaboração da redação preliminar do trabalho (Qualificação) &  & * & * & * &  &  \\ \hline
		Qualificação &  &  &  & * &  &  \\ \hline
		Extração e tratamento de dados &  &  & * & * & * &  \\ \hline
		Disseminação da pesquisa por meio da produção de artigos &  &  & * & * & * & * \\ \hline
		Produção de relatórios técnicos para FAPESP &  & * &  & * &  & * \\ \hline
		Término da escrita, entrega da tese &  &  &  &  & * & * \\ \hline
		Defesa da tese &  &  &  &  &  & * \\ \hline
	\end{tabular}
\end{table}


%%%%%%%%%%%%%%%%%%%%%%%%%%%%%%%%%%%%%% Material e Métodos %%%%%%%%%%%%%%%%%%%%%%%%%%%%%%%%%%%%%%%%%%%%%%%%%%%
\section{Material e Métodos}

Descreva os métodos e materiais que serão utilizados (1-2 pg.)

%%%%%%%%%%%%%%%%%%%%%%%%%%%%%%%%%%%%%% Forma de análise dos resultados %%%%%%%%%%%%%%%%%%%%%%%%%%%%%%%%%%%%%%%%%%%%
\section{Forma de Análise dos Resultados}

Como os resultados podem ser verificados? (1-2pg.)





%%%%%%%%%%%%%%%%%%%%%%%%%%%%%%%%%%%%%% Conclusão %%%%%%%%%%%%%%%%%%%%%%%%%%%%%%%%%%%%%%%%%%%%%%%%%%%
\section{Considerações Finais}


\newpage
\section*{Sobre o beneficiário}


\subsubsection*{Artigos apresentados}


%%%%%%%%%%%%%%%%%%%%%%%%%%%%%%%%%%%%%% Referências bibliográficas %%%%%%%%%%%%%%%%%%%%%%%%%%%%%%%%%%%%%%%%%%%%%%%%%%%
\newpage
% Trocando título da seção Referências
\renewcommand\refname{\vskip -2cm}
\section*{Referências Bibliográficas}
\bibliography{mybibfile}
\newpage
\newpage

%%%%%%%%%%%%%%%%%%%%%%%%%%%%%%%%%%%%%%%%%%%%%%%%%%%%%%%%%%%%%%%%%%%%%%%%%%%%%%%%%%%%%%%%%%%%%%%%%%%%%%%%%%%%%%%%
\end{document}          
